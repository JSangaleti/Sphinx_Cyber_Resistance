%% Generated by Sphinx.
\def\sphinxdocclass{report}
\documentclass[letterpaper,10pt,portuges]{sphinxmanual}
\ifdefined\pdfpxdimen
   \let\sphinxpxdimen\pdfpxdimen\else\newdimen\sphinxpxdimen
\fi \sphinxpxdimen=.75bp\relax
\ifdefined\pdfimageresolution
    \pdfimageresolution= \numexpr \dimexpr1in\relax/\sphinxpxdimen\relax
\fi
%% let collapsible pdf bookmarks panel have high depth per default
\PassOptionsToPackage{bookmarksdepth=5}{hyperref}

\PassOptionsToPackage{booktabs}{sphinx}
\PassOptionsToPackage{colorrows}{sphinx}

\PassOptionsToPackage{warn}{textcomp}
\usepackage[utf8]{inputenc}
\ifdefined\DeclareUnicodeCharacter
% support both utf8 and utf8x syntaxes
  \ifdefined\DeclareUnicodeCharacterAsOptional
    \def\sphinxDUC#1{\DeclareUnicodeCharacter{"#1}}
  \else
    \let\sphinxDUC\DeclareUnicodeCharacter
  \fi
  \sphinxDUC{00A0}{\nobreakspace}
  \sphinxDUC{2500}{\sphinxunichar{2500}}
  \sphinxDUC{2502}{\sphinxunichar{2502}}
  \sphinxDUC{2514}{\sphinxunichar{2514}}
  \sphinxDUC{251C}{\sphinxunichar{251C}}
  \sphinxDUC{2572}{\textbackslash}
\fi
\usepackage{cmap}
\usepackage[T1]{fontenc}
\usepackage{amsmath,amssymb,amstext}
\usepackage{babel}



\usepackage{tgtermes}
\usepackage{tgheros}
\renewcommand{\ttdefault}{txtt}



\usepackage[Sonny]{fncychap}
\ChNameVar{\Large\normalfont\sffamily}
\ChTitleVar{\Large\normalfont\sffamily}
\usepackage{sphinx}

\fvset{fontsize=auto}
\usepackage{geometry}


% Include hyperref last.
\usepackage{hyperref}
% Fix anchor placement for figures with captions.
\usepackage{hypcap}% it must be loaded after hyperref.
% Set up styles of URL: it should be placed after hyperref.
\urlstyle{same}

\addto\captionsportuges{\renewcommand{\contentsname}{Conteúdos:}}

\usepackage{sphinxmessages}
\setcounter{tocdepth}{1}



\title{Cyber Resistance}
\date{03 out., 2024}
\release{2025}
\author{UTFPR}
\newcommand{\sphinxlogo}{\vbox{}}
\renewcommand{\releasename}{Versão}
\makeindex
\begin{document}

\ifdefined\shorthandoff
  \ifnum\catcode`\=\string=\active\shorthandoff{=}\fi
  \ifnum\catcode`\"=\active\shorthandoff{"}\fi
\fi

\pagestyle{empty}
\sphinxmaketitle
\pagestyle{plain}
\sphinxtableofcontents
\pagestyle{normal}
\phantomsection\label{\detokenize{index::doc}}

\begin{itemize}
\item {} 
\sphinxAtStartPar
\DUrole{xref,std,std-ref}{search}

\end{itemize}

\sphinxstepscope


\chapter{Sobre nosso projeto}
\label{\detokenize{about:sobre-nosso-projeto}}\label{\detokenize{about::doc}}

\section{Resumo}
\label{\detokenize{about:resumo}}
\sphinxAtStartPar
A área de Segurança Cibernética, ou Cibersegurança, está em crescente destaque devido ao seu papel na proteção dos sistemas e infraestruturas digitais das ameaças cibernéticas. No entanto, formar profissionais nessa área é desafiador, visto que exige conhecimentos avançados em muitas outras áreas e dedicação contínua. Por consequência, muitos estudantes iniciam os estudos e acabam desistindo, seja por dificuldade de aprendizado ou engajamento. Assim, uma abordagem promissora para abordar essas questões é o uso de jogos eletrônicos educacionais.

\sphinxAtStartPar
Esses jogos podem ser usados para manter a motivação dos estudantes, provendo um ambiente que incentive o aprendizado por meio da superação de desafios e competitividade. Já existem vários desses jogos direcionados à cibersegurança, no entanto, apresentam\sphinxhyphen{}se mais como ferramentas para aperfeiçoamento em diferentes tópicos na área. Considerando a importância da cibersegurança na sociedade atual e a necessidade de formar profissionais qualificados na área, propõe\sphinxhyphen{}se nesta pesquisa o desenvolvimento de um jogo eletrônico em cibersegurança que aborda os aspectos educacionais e as etapas de formação de um profissional em cibersegurança.

\sphinxAtStartPar
Sendo assim, objetiva\sphinxhyphen{}se o desenvolvimento de uma plataforma gamificada para o ensino e aprendizado de segurança cibernética, considerando os aspectos educacionais por meio de um processo linear e didático. Dessa forma, espera\sphinxhyphen{}se manter os alunos engajados e motivados pelo enredo e desafios propiciados pelos jogos eletrônicos, e, ao mesmo tempo, formar profissionais qualificados na área de segurança cibernética.

\sphinxAtStartPar
Pretende\sphinxhyphen{}se adotar uma metodologia investigativa para identificar as principais formas de gamificar atividades em cibersegurança e usar tecnologias atuais para prover um ambiente gráfico em conjunto de um enredo que engaje os alunos nos estudos. Como resultados, espera\sphinxhyphen{}se obter um protótipo funcional de jogo educacional, que possibilite adicionar novos módulos educacionais a partir de novas pesquisas ou projetos de desenvolvimento tecnológico.

\sphinxAtStartPar
\sphinxhref{https://docs.google.com/document/d/1l00M8q8OG\_o0MDN-NuUi-8v9xy-QkaR2WW377ktjONE/edit?usp=sharing}{Veja o artigo completo aqui!}

\sphinxstepscope


\chapter{Tecnologias utilizadas}
\label{\detokenize{techs:tecnologias-utilizadas}}\label{\detokenize{techs::doc}}

\section{Godot}
\label{\detokenize{techs:id1}}
\sphinxAtStartPar
A \sphinxhref{https://godotengine.org/}{Godot} é um motor (ou \sphinxstyleemphasis{engine}) e plataforma \sphinxstyleemphasis{Open\sphinxhyphen{}Source} de desenvolvimento de jogos, seja 2D ou 3D (no caso da versão 4). Os códigos nela escritos são feitos na linguagem GDScript, cuja sintáxe é similar ao Python 3.


\section{Docker}
\label{\detokenize{techs:id2}}
\sphinxAtStartPar
O Docker

\sphinxstepscope


\chapter{Revisão de Literatura}
\label{\detokenize{lit_rev:revisao-de-literatura}}\label{\detokenize{lit_rev::doc}}

\section{OverTheWire}
\label{\detokenize{lit_rev:id1}}
\sphinxAtStartPar
\sphinxhref{https://overthewire.org/wargames/}{OverTheWire} é um site educacional que providencia alguns "jogos" (que na verdade são exercícios de linguagem gamificados) de forma completamente gratuita. A ênfase do ensino desta plataforma é em cibersegurança. Contudo, também há modos cujo objetivo de desenvolvimento está em comandos Shell (linguagem utilizada nos emuladores de terminal Linux).

\sphinxAtStartPar
É um excelente material para todos os níveis de estudantes ou amantes da computação, principalmente porque referencia as ferramentas necessárias para resolver os desafios, que, por sinal, são construídos de forma escalar, guiando o jogador por um caminho de conhecimento ao longo da trajetória determinada pela categoria.

\sphinxstepscope


\chapter{Contate\sphinxhyphen{}nos!}
\label{\detokenize{contact:contate-nos}}\label{\detokenize{contact::doc}}

\section{Whatsapp}
\label{\detokenize{contact:whatsapp}}

\section{Email}
\label{\detokenize{contact:email}}
\sphinxstepscope


\chapter{Apoie\sphinxhyphen{}nos!}
\label{\detokenize{support:apoie-nos}}\label{\detokenize{support::doc}}

\section{Nossas redes sociais}
\label{\detokenize{support:nossas-redes-sociais}}
\sphinxAtStartPar
{\color{red}\bfseries{}\textasciigrave{}Instagram \textless{}\textgreater{}\textasciigrave{}\_}


\section{Projetos parecidos}
\label{\detokenize{support:projetos-parecidos}}

\section{Doações}
\label{\detokenize{support:doacoes}}


\renewcommand{\indexname}{Índice}
\printindex
\end{document}